\documentclass[10pt]{article}
\usepackage{calc}
\usepackage[usenames,dvipsnames]{xcolor}
\definecolor{lightgray}{gray}{0.9}

\usepackage{tikz}
\usetikzlibrary{calc}
\usepackage{xparse}
\usepackage{etoolbox}
\usepackage[graphics, tightpage, active]{preview}
\usepackage{amsmath}


\usetikzlibrary{decorations.pathreplacing}


\setlength{\PreviewBorder}{2pt}
\PreviewEnvironment{tikzpicture}


%============================================================================================
% Math symbols
\newcommand \Rbb{\mathbb{R}}
%\newcommand \coloneqq{:=} defined in mathtools package
\newcommand \supp{\textup{supp}}
\newcommand \Supp[1]{ \supp \left( #1 \right) }
\newcommand \eqDef{\overset{\Delta}{=}}

\newcommand \vect[1]{\boldsymbol #1}
\newcommand \qed {\begin{flushright} $\blacksquare$ \end{flushright}}
%============================================================================================
% Variables 
\newcommand \flow{x}
\newcommand \flowV{\vect \flow}
\newcommand \flowMax{\flow^{\max}}
\newcommand \mode{m}
\newcommand \modeV{\vect \mode}

\newcommand \mass{\rho}
\newcommand \massMax{\mass^{\max}}
\newcommand \massCrit{\mass^{\text{crit}}}

\newcommand \speedFF{v^{f}}
\newcommand \speedCong{v^{c}}

%============================================================================================
% Parameters
\newcommand \length{L}
\newcommand \NLinks{N}
\newcommand \compRate{\alpha}
\newcommand \cR{\compRate}
\newcommand \demand{r}
\newcommand \demandMax[1]{\demand^{\textup{NE}} ( #1 ) }
\newcommand \parA{a}

%============================================================================================
% Notation
\newcommand \link{n}

% Stackelberg thresholds notation
\newcommand \lastNash[1]{b(#1)}
\newcommand \lastStack[1]{l(#1)}
\newcommand \lastNC[1]{k(#1)}
\newcommand \selfish[1]{\demand_{#1}}
\newcommand \cardNash{c}
\newcommand \maxR[1]{\demandMax{k_{#1}}}
% 

\newcommand \cFlow[2]{ \hat{\flow}_{#1} ( #2 ) }
\newcommand \ffNashMode[1]{ \modeV^{#1}}
\newcommand \ffNashFlow[2]{\flowV^{#1,#2}}

\newcommand \stackFlow{s}
\newcommand \sV{\vect \stackFlow}
\newcommand \respFlow{t}
\newcommand \tV{\vect \respFlow}
\newcommand \epsV{\vect \epsilon}


\newcommand \latency{\ell}
\newcommand \latencyMass{\latency^{\mass}}
\newcommand \totalLatency{C}
\newcommand \massCongFunc{\mass^{\text{cong}}}

\newcommand \BNE[2]{\textup{BNE} (#1, #2) }
\newcommand \NE[2]{\textup{NE} (#1, #2) }
\newcommand \NEf[2]{\textup{NE}_\textup{f} (#1, #2) }
\newcommand \NEc[2]{\textup{NE}_\textup{c} (#1, #2) }
\newcommand \NCF[3]{\textup{NCF} (#1, #2, #3) }
\newcommand \stackSet[3]{\textup{S} (#1, #2, #3) }
\newcommand \stackSetOpt[3]{\textup{S}^\star (#1, #2, #3) }
\newcommand \POS[3]{\textup{POS} (#1,#2, #3)}
\newcommand \SO[2]{\textup{SO} (#1,#2)}

\newcommand \std[1] {#1{\NLinks}{\demand}}
\newcommand \stdStack[1]{#1{\NLinks}{\demand}{\compRate}}


%============================================================================================
% over braces and under braces
\usepackage{mathtools}

\newcommand \overbrac[3]{
\overbracket{\vphantom{#1} #2}^{\mathclap{#3}}
}
\newcommand \underbrac[3]{
\underbracket{\vphantom{#1} #2}_{\mathclap{#3}}
}

\newcommand \overbracBig[2]{
\overbrac{\Big(}{#1}{#2}
}
\newcommand \underbracBig[2]{
\underbrac{\Big(}{#1}{#2}
}

\newcommand \overbracbig[2]{
\overbrac{\big(}{#1}{#2}
}
\newcommand \underbracbig[2]{
\underbrac{\big(}{#1}{#2}
}

%-----------------------------------------------------------------------------------------------------------------------------------------------------------------
% some definitions for the table
\newcommand{\vcell}[2][c]{
\begin{tabular}[t]{@{}p{#1}@{}}
#2
\end{tabular}
}

\def \bul{$\centerdot$ }

%-----------------------------------------------------------------------------------------------------------------------------------------------------------------



%============================================================================================

\newcommand \trim[4]{
\renewcommand \PreviewBbAdjust{#1 #2 #3 #4}
}

% global variables and commands
\def \latencyRange{1}
\def \flowRange{1}
\def \massRange{1}

\newcommand \LineDashedVer[3]{
[dashed] (#1,#2) -- (#1,0) node[below] {#3};
}
\newcommand \LineDashedHor[3]{
[dashed] (#1,#2) -- (0,#2) node[left] {#3};
}
\newcommand \BraceHorAbove[4]{
[decorate,decoration=brace,yshift=2pt] (#1,#3) -- (#2,#3)node[above,midway]{#4}
}

%-----------------------------------------------------------------------------------------------------------------------------------------------------------------
\newcommand \NWHashedRectangle[4]{
 (#1,#2) rectangle (#3,#4);
\begin{scope}
\path (#1,#2) coordinate (c0);
\path (#3,#4) coordinate (c1);
\path (-\latencyRange,0) coordinate (p0);
\path (\flowRange+\latencyRange,0) coordinate (p1);
\clip (c0) rectangle (c1);
\draw [rotate around=(45:($(c0)!0.5!(c1)$)),xstep=0.1cm,ystep=2cm,yshift=-5cm] (p0) grid (p1);
\end{scope}
}

\newcommand \NEHashedRectangle[4]{
 (#1,#2) rectangle (#3,#4);
\begin{scope}
\path (#1,#2) coordinate (c0);
\path (#3,#4) coordinate (c1);
\path (-\latencyRange,0) coordinate (p0);
\path (\flowRange+\latencyRange,0) coordinate (p1);
\clip (c0) rectangle (c1);
\draw [rotate around=(-45:($(c0)!0.5!(c1)$)),xstep=0.1cm,ystep=10cm,yshift=-5cm] (p0) grid (p1);
\end{scope}
}


%-----------------------------------------------------------------------------------------------------------------------------------------------------------------
% Triangular
% Length is assumed to be 1. The parameters are:
% -free-flow speed
% -critical density
% -jam density
\newcommand{\drawHQTriangularFlux}[3]{
\draw (0,0) -- (#2,#1*#2) -- (#3, 0)node[below]{$\massMax_n$};
\draw\LineDashedHor{#2}{#1*#2}{$\massCrit_n$};
\draw\LineDashedVer{#2}{#1*#2}{$\flowMax_n$};
}
\newcommand \drawHQTriangularDensityLat[3]{
\draw (0,1/#1)node[left]{$\parA_n$} -- +(#2,0);
\draw\LineDashedVer{#2}{1/#1}{$\massCrit_n$};
\draw\LineDashedVer{#3}{\latencyRange}{$\massMax_n$};
\begin{scope} %scope for clipping (cropping) the function
\clip (0,0) rectangle (\massRange, \latencyRange); 
\draw plot[domain=#2:#3] function{x/(x-#3)*(#2-#3)/(#1*#2)};
\end{scope} 
}

\newcommand{\drawHQTriangularLat}[3]{
\drawHQTriangularLatFunc{#1}{#2}{#3}
\drawHQTriangularLatFlowMax{#1}{#2}{#3}
}

%------------------------------------------------------------------------------
\NewDocumentCommand \drawHQTriangularLatFunc{mmmD<>{n}}{
\draw (0,1/#1)node[left]{$\parA_#4$} -- (#1*#2,1/#1);
\begin{scope} 
\clip (0,0) rectangle (\flowRange, \latencyRange); 
\draw plot[domain=0:#1*#2] function{#3/x + 1.*(#2-#3)/(#1*#2)}; 
\end{scope}
}
\NewDocumentCommand{\drawHQTriangularLatFlowMax}{mmmD<>{n}}{
\draw\LineDashedVer{#1*#2}{1/#1}{$\flowMax_#4$};
}

%------------------------------------------------------------------------------
\NewDocumentCommand\drawcFlow{mmmo}{
\newdimen \tmp; \tmp=#1cm;
\ifdimcomp{\ffLat\tmp}{>}{1.01cm}
{
\draw ({\triangularLatInvert{#1}{#2}{#3}{\ffLat}}, \ffLat) circle (2pt);
\IfNoValueTF{#4}{}{\draw\LineDashedVer{{\triangularLatInvert{#1}{#2}{#3}{\ffLat}}}{\ffLat}{#4}}
}{};
}
\newcommand{\triangularLatInvert}[4]{%4rth parameter is the value to invert
#3/(#4-1.*(#2-#3)/(#1*#2))
}


