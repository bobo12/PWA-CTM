\section{Introduction}

Numerous traffic estimation techniques developed in the literature rely on density based traffic models such as the Lighthill-Whitham-Richards (LWR) partial differential equation (PDE) \cite{Lighthill1955,Richards1956} and its discretization using the Godunov scheme \cite{Lebacque1996,LeVeque1992,Strub2006} (also known as the Cell Transmission Model (CTM) \cite{Daganzo1994,Daganzo1995} in the transportation literature). These highway traffic monitoring systems rely on large amounts of data from different sources. These include \textit{inductive loop detectors} (ILD) used in the PeMS system \cite{Chen2005} and \textit{in-vehicle transponders} (IVTs) such as FasTrak. Recently, the available data on traffic has increased tremendously since the development of cellular phone based highway traffic monitoring. With the cellular phone communication infrastructure in place and privacy aware smartphone sensing technology in full expansion \cite{Hoh2008}, a large volume of data from mobile devices is now available \cite{Herrera2009}. Large scale Applications include traffic flow estimation to assimilate velocity measurements \cite{Work2008,Work2008a}, which is a rapidly expanding field at the heart of mobile internet services. This points out on the necessity of powerful statistical filters and algorithms to efficiently assimilate the measurements.

In \cite{Work2008,Work2008a} the \textit{Ensemble Kalman Filter} is used to assimilate velocity measurements. In \cite{Munoz2003}, a switching-mode model (SMM) has been derived from the CTM, which is a nonlinear discrete time dynamical system. This consists in switching among different sets of linear difference equations, defined as linear state-space model (SSM) or modes, combined with a hidden Markov model to describe the transitions from one mode to another. The Mixture Kalman filter algorithm \cite{Chen2000} is employed to assimilate data in a switching state-space model. In this paper, we show that for a Daganzo-Newell fundamental diagram, the Godunov scheme applied to the LWR model (described in \cite{Daganzo1995}) is a piecewise affine (PWA) dynamic system, where each affine component is a linear mode. Contrary to the SMM, where an additional statistical model, namely the hidden Markov model, is introduced, we unravel the PWA character of the original CTM.